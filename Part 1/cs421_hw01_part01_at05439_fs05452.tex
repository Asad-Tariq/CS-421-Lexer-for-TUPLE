\documentclass[addpoints]{exam}

\usepackage{graphbox}
\usepackage{hyperref}
\usepackage{listings}
\usepackage{tabularx}
\usepackage{tikz}
\usepackage{float}
\usepackage{enumitem}
\usepackage{setspace}
\usetikzlibrary{positioning}
\usepackage{xcolor}

% Header and footer.
\pagestyle{headandfoot}
\runningheadrule
\runningfootrule
\runningheader{CS 421}{HW 1: Lexical Analyzer}{Fall 2022}
\runningfooter{}{Page \thepage\ of \numpages}{}
\firstpageheader{}{}{}

\boxedpoints
\printanswers

\definecolor{codegreen}{rgb}{0,0.6,0}
\definecolor{codegray}{rgb}{0.5,0.5,0.5}
\definecolor{codepurple}{rgb}{0.58,0,0.82}
\definecolor{backcolour}{rgb}{0.95,0.95,0.92}

\lstdefinestyle{mystyle}{
    backgroundcolor=\color{backcolour},   
    commentstyle=\color{codegreen},
    keywordstyle=\color{magenta},
    numberstyle=\tiny\color{codegray},
    stringstyle=\color{codepurple},
    basicstyle=\ttfamily\footnotesize,
    breakatwhitespace=false,         
    breaklines=true,                 
    captionpos=b,                    
    keepspaces=true,                 
    numbers=left,                    
    numbersep=5pt,                  
    showspaces=false,                
    showstringspaces=false,
    showtabs=false,                  
    tabsize=2
}

\lstset{style=mystyle}


\title{Homework 1: Lexical Analyzer\\CS 421 Compiler Design and Construction}
\author{Asad Tariq \& Fahad Shaikh}
\date{Habib University\\Fall 2022\\\textbf{\textcolor{red}{Due:}} September 11, 2022}

\begin{document}
\maketitle

\noindent

\begin{questions}

  \question[1]
  Consider the following \texttt{C} code snippet:
  \begin{lstlisting}[language=C, caption={\texttt{C} code snippet.}, label={lst:firstSnip}]
    int w, x, y, z;
    int i = 4; int j = 5;
    {
        int j = 7;
        i = 6;
        w = i + j;
    }
    x = i + j;
    {
        int i = 8;
        y = i + j;
    }
    z = i + j;
  \end{lstlisting}
  Indicate the values assigned to \textbf{\texttt{x}} and \textbf{\texttt{z}} from Listing~\ref{lst:firstSnip}.

  \begin{solution}
    On examination of the given \texttt{C} code in Listing~\ref{lst:firstSnip}, the values of \texttt{x} and \texttt{z} are as follows:
    \begin{itemize}
      \item \texttt{x} = 11
      \item \texttt{z} = 11
    \end{itemize}
  \end{solution}

  \question[1]
  What is printed as output when the following \texttt{C} code is executed?
  \begin{lstlisting}[language=C, caption={\texttt{C} code snippet.}, label={lst:secSnip}]
    #define a (x+1)
    int x = 2;
    void b() {x = a; printf("%d\n", x);}
    void c() {int x = 1; printf("%d\n", a);}
    void main() {b(); c();}
  \end{lstlisting}

  \begin{solution}
    The output of the code in Listing~\ref{lst:secSnip} is as follows:\\3\\2
  \end{solution}

  \question[1]
  Construct a regular expression to recognize \textit{currency} numbers in dollars.
  It should be a positive decimal number rounded to the nearest one-hundredth.
  Currency numbers begin with the dollar sign \$, have commas separating each group of three digits to the left of the decimal point, and end with two digits to the right of the decimal point, for example, \$8,937.43 and \$7,777,777.7.
  
  \begin{solution}
    For construction of the required regular expression, we shall be relying on Table~\ref{tab:Table 1}.
    \begin{table}[H]
      \begin{center}
          \begin{tabular}{ |c|c|c| } 
              \hline
              \hline
              EXPRESSION & MATCHES & EXAMPLE \\ 
              \hline
              \^{} & beginning of a line & \^{}\texttt{abc} \\
              \hline
              \(c\) & the one non-operator character \(c\) & , \\
              \hline
              \textbackslash \(c\) & character \(c\) literally & \textbackslash \$ \\
              \hline 
              \$ & end of a line & \texttt{abc}\$ \\
              \hline
              \textbackslash \(d\) & digit \(d\) & 5 \\
              \hline 
              \(r\)\{m, n\} & between \(m\) and \(n\) occurrences of \(r\) & \texttt{a}\{3, 5\} \\
              \hline
              \(r\)\{n\} & strictly \(n\) occurrences of \(r\) & \texttt{a}\{2\}\\
              \hline
              \(r\)? & zero or one \(r\) & \texttt{a}? \\
              \hline
              \(r\)* & zero or many strings matching \(r\) & \texttt{a}* \\
              \hline
          \end{tabular}
      \end{center}
      \caption{\label{tab:Table 1}A modification of the `\texttt{Lex} regular expressions' table from Section 3.3 (Figure 3.8) of the \textit{Dragon Book}.}
  \end{table}
  Now, using Table~\ref{tab:Table 1}, we construct the following regular expression:
  
  \begin{center}
    \^{}\textbackslash\$\textbackslash \(d\)\{1, 3\}(,\textbackslash \(d\)\{3\})*(\textbackslash .\textbackslash \(d\)\textbackslash \(d\))\$
  \end{center}
  
  \underline{Breakdown}:
  \begin{itemize}
    \item \^{}\textbackslash\$: The string should start with the `\$' character.
    \item \textbackslash \(d\)\{1, 3\}: One, two or three digits.
    \item (,\textbackslash \(d\)\{3\})*: A group of characters - a comma followed by exactly three digits - that may appear zero or many times.
    \item (\textbackslash .\textbackslash \(d\)\textbackslash \(d\))\$: The final (i.e., \textit{ending}) portion of the string should be a decimal point followed by two digits.
  \end{itemize}
\end{solution}
\end{questions}

\end{document}
%%% Local Variables:
%%% mode: latex
%%% TeX-master: t
%%% End:

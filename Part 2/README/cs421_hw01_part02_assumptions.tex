\documentclass[a4paper, 10pt, oneside]{article}

\usepackage[utf8]{inputenc}
\usepackage[english]{babel}
\usepackage{amsmath}
\usepackage{tikz}
\usepackage{hyperref}
\usepackage{setspace}
\usepackage[a4paper, left=1in,right=1in,top=1in,bottom=1in]{geometry}

\title{Homework 1: Lexical Analyzer \footnote{This document is with reference to Part 2 of the assignment.}\\CS 421 Compiler Design and Construction}
\author{\href{mailto:at05439@st.habib.edu.pk}{Asad Tariq} \& \href{mailto:fs05452@st.habib.edu.pk}{Fahad Shaikh}}
\date{Habib University\\Fall 2022\\\textbf{\textcolor{red}{Due:}} September 11, 2022}


\onehalfspacing

\begin{document}

\maketitle
\noindent
With reference to the \textbf{Lexical Analyzer} implemented for the language \texttt{\textcolor{blue}{TUPLE}}, let us, \textit{the programmers},
elucidate some of the assumptions taken, conventions followed and references used. They are as follows:
\begin{itemize}
    \item Some modifications were made to the given lexical specification - specified on \emph{\textbf{pg.2}} of the assignment write-up. These changes
    are with respect to the \textbf{Keyword} and \textbf{Data Types} categories - upon examination of the test files, it was seen that the common language
    keywords and data types, \texttt{print}, \texttt{return} and \texttt{str} were not specified within the lexical specification thus, they were catered to
    as such per our own prerogative.
    \item With reference to entries within the symbol table, the decision was made not to include numeric constants such as: \texttt{2.3}, \texttt{-5}, \texttt{10} etc. Per the
    current implementation of lexical analyzer, the symbol table only records identifiers.
    \item As of now, all arithmetic operations are dealt with as part of a single category. Delineation of arithmetic operators with respect to precedence\footnote{As discussed in the lecture session of $9^{th}$ September.} is not implemented.
    \item This \href{https://austinhenley.com/blog/teenytinycompiler1.html}{reference} proved useful when starting out with implementation of the analyzer.
\end{itemize}

\end{document}